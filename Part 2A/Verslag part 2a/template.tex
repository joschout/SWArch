\documentclass[a4paper,10pt]{article}

\usepackage[english]{babel}
\usepackage{graphicx}
\graphicspath{{./figures/}}
\usepackage[colorlinks, linkcolor=black, citecolor=black, urlcolor=black]{hyperref}
\usepackage{geometry}
\geometry{tmargin=3cm, bmargin=2.2cm, lmargin=2.2cm, rmargin=2cm}
\usepackage{todonotes} %Used for the figure placeholders

\begin{document}
\begin{titlepage}
    \newpage
    \thispagestyle{empty}
    \frenchspacing
    \hspace{-0.2cm}
    \includegraphics[height=3.4cm]{sedes}
    \hspace{0.2cm}
    \rule{0.5pt}{3.4cm}
    \hspace{0.2cm}
    \begin{minipage}[b]{8cm}
        \Large{Katholieke\newline Universiteit\newline Leuven}\smallskip\newline
        \large{}\smallskip\newline
        \textbf{Department of\newline Computer Science}\smallskip
    \end{minipage}
    \hspace{\stretch{1}}
    \vspace*{3.2cm}\vfill
    \begin{center}
        \begin{minipage}[t]{\textwidth}
            \begin{center}
                \LARGE{\rm{\textbf{\uppercase{Document Processing}}\\Domain 
                Analysis}}\\
                \Large{\rm{Software Architecture (H09B5a and H07Z9a) -- 
                Part 1}}
            \end{center}
        \end{minipage}
    \end{center}
    \vfill
    \hfill\makebox[8.5cm][l]{%
        \vbox to 7cm{\vfill\noindent
            {\rm \textbf{Student A (r123456)}}\\
            {\rm \textbf{Student B (r987654)}}\\[2mm]
            {\rm Academic year 2014--2015}
        }
    }
\end{titlepage}


\tableofcontents
\newpage

\section{Introduction}\label{sec:introduction}

\section{Attribute-driven design documentation}\label{sec:add}
\subsection{Decomposition 1: ModuleA (X1, Y3, UCa, UCb, UCc)}
\subsubsection{Module to decompose}
In this run we decompose \texttt{ModuleA}.

\subsubsection{Selected architectural drivers}
The non-functional drivers for this decomposition are:

\begin{itemize}
    \item \emph{X1}: name
    \item \emph{Y3}: name
\end{itemize}

The related functional drivers are:

\begin{itemize}
    \item \emph{UCa}: name
    \item \emph{UCb}: name
    \item \emph{UCc}: name
\end{itemize}

\paragraph{Rationale}
A short discussion of why these drivers were selected for this decomposition.

\subsubsection{Architectural design}
\paragraph{Topic}
Discussion of the solution selected for (a part of) one of the architectural
drivers.

\subsubsection*{Alternatives considered}
\paragraph{Alternatives for solution}
A discussion of the alternative solutions and why that were not selected.

\subsubsection{Instantiation and allocation of functionality}
\paragraph{Decomposition}
Main aspects of the resulting decomposition.

\subparagraph{ModuleB}
Per introduced component a paragraph describing its responsibilities.

\subparagraph{ModuleC}
Per introduced component a paragraph describing its responsibilities.

\begin{figure}[!htp]
    \centering
    %\includegraphics[width=0.8\textwidth]{}
    \missingfigure[figwidth=0.8\textwidth]{Component-and-connector diagram}
    \caption{Component-and-connector diagram of this decomposition.
        }\label{fig:it1-cc_main}
\end{figure}

\paragraph{Behaviour}
If needed and explanation of the behaviour of certain aspects of the design so
far.

\begin{figure}[!htp]
    \centering
    %\includegraphics[width=0.8\textwidth]{}
    \missingfigure[figwidth=0.8\textwidth]{Sequence diagram}
    \caption{Sequence diagram illustrating a key behavioural aspect.
        }\label{fig:it1-seq_aspect1}
\end{figure}

\paragraph{Deployment}
Rationale of the allocation of components to physical nodes.

\begin{figure}[!htp]
    \centering
    %\includegraphics[width=0.8\textwidth]{}
    \missingfigure[figwidth=0.8\textwidth]{Deployment diagram}
    \caption{Deployment diagram of this decomposition.
        }\label{fig:it1-depl_main}
\end{figure}

\subsubsection{Interfaces for child modules}
\subsubsection*{ModuleB}
\begin{itemize}
    \item InterfaceA
    \begin{itemize}
        \item \texttt{returnType operation1(ParamType param1)} throws TypeOfException
        \begin{itemize}
            \item Effect: Describe the effect of calling this operation.
            \item Exceptions: 
            \begin{itemize}
                \item TypeOfException: Describe when this exception is thrown.
            \end{itemize}
        \end{itemize}

        \item \texttt{returnType operation2()}
        \begin{itemize}
            \item Effect: Describe the effect of calling this operation.
            \item Exceptions: None
         \end{itemize}
    \end{itemize}
\end{itemize}

\subsubsection{Data type definitions}
Describe per complex data type used in the interfaces what it represents.

\paragraph{returnType} This data element represents X.

\paragraph{ParamType} This data element represents Y.

\subsubsection{Verify and refine}
This section describes per component which (parts of) the remaining
requirements it is responsible for.

\paragraph{ModuleB}
\begin{itemize}
    \item \emph{Z1}: name
    \item \emph{UCd}: name
\end{itemize}

\paragraph{ModuleC}
\begin{itemize}
    \item \emph{UCba}: name\\Description which part of the original use case is
        the responsibility of this component.
\end{itemize}

\subsection{Decomposition 2: Module (drivers)}
\subsubsection{Module to decompose}
\subsubsection{Selected architectural drivers}
\subsubsection{Architectural design}
\subsubsection{Instantiation and allocation of functionality}
\subsubsection{Interfaces for child modules}
\subsubsection{Data type definitions}
\subsubsection{Verify and refine}

\subsection{Decomposition 3: Module (drivers)}
\subsubsection{Module to decompose}
\subsubsection{Selected architectural drivers}
\subsubsection{Architectural design}
\subsubsection{Instantiation and allocation of functionality}
\subsubsection{Interfaces for child modules}
\subsubsection{Data type definitions}
\subsubsection{Verify and refine}

\section{Resulting partial architecture}\label{sec:architecture}
This section provides an over of the architecture constructed through ADD\@.

\subsection{Context diagram}
This subsection discusses the context diagram.

\begin{figure}[!htp]
    \centering
    %\includegraphics[width=0.8\textwidth]{}
    \missingfigure[figwidth=0.8\textwidth]{Context diagram for component-and-
        connector view.}
    \caption{Context diagram for the component-and-connector view.
        }\label{fig:cc_context}
\end{figure}

\subsection{Component-and-connector view}
A short discussion of the component-and-connector view with the key
decompositions if any.

\begin{figure}[!htp]
    \centering
    %\includegraphics[width=0.8\textwidth]{}
    \missingfigure[figwidth=0.8\textwidth]{Component-and-connector diagram}
    \caption{Primary diagram for the component-and-connector view.
        }\label{fig:cc_main}
\end{figure}

\begin{figure}[!htp]
    \centering
    %\includegraphics[width=0.8\textwidth]{}
    \missingfigure[figwidth=0.8\textwidth]{Key decomposition}
    \caption{Decomposition of a component shown in Figure~\ref{fig:cc_main}
        }\label{fig:decomp_decomp1}
\end{figure}

\subsection{Deployment view}
A short discussion of the allocation of components to physical nodes based on a
context diagram and a deployment diagram.

\begin{figure}[!htp]
    \centering
    %\includegraphics[width=0.8\textwidth]{}
    \missingfigure[figwidth=0.8\textwidth]{Context diagram for the allocation
        view.}
    \caption{Context diagram for the allocation view.}\label{fig:depl_context}
\end{figure}

\begin{figure}[!htp]
    \centering
    %\includegraphics[width=0.8\textwidth]{}
    \missingfigure[figwidth=0.8\textwidth]{Deployment diagram}
    \caption{Primary diagram for the allocation view.}\label{fig:depl_main}
\end{figure}

\end{document}
