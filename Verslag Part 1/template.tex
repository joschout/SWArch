\documentclass[a4paper,10pt]{article}

\usepackage[english]{babel}
\usepackage{graphicx}
\usepackage[colorlinks, linkcolor=black, citecolor=black, urlcolor=black]{hyperref}
\usepackage{geometry}
\geometry{tmargin=3cm, bmargin=2.2cm, lmargin=2.2cm, rmargin=2cm}
\usepackage{todonotes} %Used for the figure placeholders

% Your name and student number must be filled in on the title page found in
% titlepage.tex.

\begin{document}
\begin{titlepage}
    \newpage
    \thispagestyle{empty}
    \frenchspacing
    \hspace{-0.2cm}
    \includegraphics[height=3.4cm]{sedes}
    \hspace{0.2cm}
    \rule{0.5pt}{3.4cm}
    \hspace{0.2cm}
    \begin{minipage}[b]{8cm}
        \Large{Katholieke\newline Universiteit\newline Leuven}\smallskip\newline
        \large{}\smallskip\newline
        \textbf{Department of\newline Computer Science}\smallskip
    \end{minipage}
    \hspace{\stretch{1}}
    \vspace*{3.2cm}\vfill
    \begin{center}
        \begin{minipage}[t]{\textwidth}
            \begin{center}
                \LARGE{\rm{\textbf{\uppercase{Document Processing}}\\Domain 
                Analysis}}\\
                \Large{\rm{Software Architecture (H09B5a and H07Z9a) -- 
                Part 1}}
            \end{center}
        \end{minipage}
    \end{center}
    \vfill
    \hfill\makebox[8.5cm][l]{%
        \vbox to 7cm{\vfill\noindent
            {\rm \textbf{Student A (r123456)}}\\
            {\rm \textbf{Student B (r987654)}}\\[2mm]
            {\rm Academic year 2014--2015}
        }
    }
\end{titlepage}


\tableofcontents
\newpage

\section{Domain analysis}\label{sec:domain}
\subsection{Domain models}
This section shows the domain model(s).

\begin{figure}[!htp]
    \centering
    %\includegraphics[width=0.8\textwidth]{}
    \missingfigure[figwidth=0.8\textwidth]{Domain model}
    \caption{The domain model for the system.}\label{fig:domain_model}
\end{figure}

\subsection{Domain constraints}
In this section we provide additional domain constraints.

\begin{itemize}
    \item This is a first constraint.
    \item This is a second constraint.
\end{itemize}

\subsection{Glossary}
In this section, we provide a glossary of the most important terminology used
in this analysis.

\begin{itemize}
    \item \textbf{Term1}: definition
    \item \textbf{Term2}: definition
\end{itemize}

\section{Functional requirements}\label{sec:functional}
\subsection*{Use case model}

\begin{figure}[!htp]
    \centering
    %\includegraphics[width=0.8\textwidth]{}
    \missingfigure[figwidth=0.8\textwidth]{Use case model}
    \caption{Use case diagram for the system.}\label{fig:use_case_model}
\end{figure}

\subsection{\emph{UC1}: Name of use case 1}
\begin{itemize}
    \item \textbf{Name:} Name of use case 1
    \item \textbf{Primary actor:} primary actor
    \item \textbf{Interested parties:} 
        \begin{itemize}
            \item \textit{Name of interested party:} reason why party is interested
        \end{itemize}

    \item \textbf{Preconditions:}
        \begin{itemize}
            \item First precondition.
            \item Second precondition.
        \end{itemize}

    \item \textbf{Postconditions:}
        \begin{itemize}
            \item First postcondition.
            \item Second postcondition.
        \end{itemize}
        
    \item \textbf{Main scenario:} 
    \begin{enumerate}
       \item Step 1
       \item Step 2
       \item Step 3
       \item \ldots
    \end{enumerate}

    \item \textbf{Alternative scenarios:} 
    \begin{enumerate}
        \item [3b.] Alternative at step 3
    \end{enumerate}
    
    \item \textbf{Remarks:}
        \begin{itemize}
            \item First remark
        \end{itemize}
\end{itemize}

\section{Non-functional requirements}\label{sec:non-functional}
In this section, we model the non-functional requirements for the system in the
form of \emph{quality attribute scenarios}. We provide for each type
(availability, performance and modifiability) one requirement.

\subsection{Availability}
\subsubsection{\emph{Av1}: Name of the quality attribute scenario}
Shortly describe the context of the scenario.

\begin{itemize}
    \item \textbf{Source:} source
    \item \textbf{Stimulus:}
        \begin{itemize}
            \item Description of a first stimulus.
            \item Description of a second stimulus.
        \end{itemize}

    \item \textbf{Artifact:} the stimulated artifact
    \item \textbf{Environment:} the condition under which the stimulus occurs
    \item \textbf{Response:}
        \begin{itemize}
            \item Describe how the system should respond to the stimulus.
        \end{itemize}

    \item \textbf{Response measure:}
        \begin{itemize}
            \item Describe how the satisfaction of a response is measured.
        \end{itemize}
\end{itemize}

\subsection{Performance}
\subsubsection{\emph{P1}: Name of the quality attribute scenario}
Shortly describe the context of the scenario.

\begin{itemize}
    \item \textbf{Source:} source
    \item \textbf{Stimulus:}
        \begin{itemize}
            \item Description of a first stimulus.
            \item Description of a second stimulus.
        \end{itemize}

    \item \textbf{Artifact:} the stimulated artifact
    \item \textbf{Environment:} the condition under which the stimulus occurs
    \item \textbf{Response:}
        \begin{itemize}
            \item Describe how the system should respond to the stimulus.
        \end{itemize}

    \item \textbf{Response measure:}
        \begin{itemize}
            \item Describe how the satisfaction of a response is measured.
        \end{itemize}
\end{itemize}

\subsection{Modifiability}
\subsubsection{\emph{M1}: Name of the quality attribute scenario}
Shortly describe the context of the scenario.

\begin{itemize}
    \item \textbf{Source:} source
    \item \textbf{Stimulus:}
        \begin{itemize}
            \item Description of a first stimulus.
            \item Description of a second stimulus.
        \end{itemize}

    \item \textbf{Artifact:} the stimulated artifact
    \item \textbf{Environment:} the condition under which the stimulus occurs
    \item \textbf{Response:}
        \begin{itemize}
            \item Describe how the system should respond to the stimulus.
        \end{itemize}

    \item \textbf{Response measure:}
        \begin{itemize}
            \item Describe how the satisfaction of a response is measured.
        \end{itemize}
\end{itemize}

\end{document}
