\documentclass[a4paper,10pt]{article}

\usepackage[english]{babel}
\usepackage{graphicx}
\usepackage[colorlinks, linkcolor=black, citecolor=black, urlcolor=black]{hyperref}
\usepackage{geometry}
\geometry{tmargin=3cm, bmargin=2.2cm, lmargin=2.2cm, rmargin=2cm}
\usepackage{todonotes} %Used for the figure placeholders

% Your name and student number must be filled in on the title page found in
% titlepage.tex.

\begin{document}
\begin{titlepage}
    \newpage
    \thispagestyle{empty}
    \frenchspacing
    \hspace{-0.2cm}
    \includegraphics[height=3.4cm]{sedes}
    \hspace{0.2cm}
    \rule{0.5pt}{3.4cm}
    \hspace{0.2cm}
    \begin{minipage}[b]{8cm}
        \Large{Katholieke\newline Universiteit\newline Leuven}\smallskip\newline
        \large{}\smallskip\newline
        \textbf{Department of\newline Computer Science}\smallskip
    \end{minipage}
    \hspace{\stretch{1}}
    \vspace*{3.2cm}\vfill
    \begin{center}
        \begin{minipage}[t]{\textwidth}
            \begin{center}
                \LARGE{\rm{\textbf{\uppercase{Document Processing}}\\Domain 
                Analysis}}\\
                \Large{\rm{Software Architecture (H09B5a and H07Z9a) -- 
                Part 1}}
            \end{center}
        \end{minipage}
    \end{center}
    \vfill
    \hfill\makebox[8.5cm][l]{%
        \vbox to 7cm{\vfill\noindent
            {\rm \textbf{Student A (r123456)}}\\
            {\rm \textbf{Student B (r987654)}}\\[2mm]
            {\rm Academic year 2014--2015}
        }
    }
\end{titlepage}


\tableofcontents
\newpage

\section{Domain analysis}\label{sec:domain}
\subsection{Domain models}
This section shows the domain model(s).

\begin{figure}[!htp]
    \centering
    %\includegraphics[width=0.8\textwidth]{}
    \missingfigure[figwidth=0.8\textwidth]{Domain model}
    \caption{The domain model for the system.}\label{fig:domain_model}
\end{figure}

\subsection{Domain constraints}
In this section we provide additional domain constraints.

\begin{itemize}
    \item This is a first constraint.
    \item This is a second constraint.
\end{itemize}

\subsection{Glossary}
In this section, we provide a glossary of the most important terminology used
in this analysis.

\begin{itemize}
    \item \textbf{Term1}: definition
    \item \textbf{Term2}: definition
\end{itemize}

\section{Functional requirements}\label{sec:functional}
\subsection*{Use case model}

\begin{figure}[!htp]
    \centering
    %\includegraphics[width=0.8\textwidth]{}
    \missingfigure[figwidth=0.8\textwidth]{Use case model}
    \caption{Use case diagram for the system.}\label{fig:use_case_model}
\end{figure}

\subsection{\emph{UC1}: Name of use case 1}
\begin{itemize}
    \item \textbf{Name:} Name of use case 1
    \item \textbf{Primary actor:} primary actor
    \item \textbf{Interested parties:} 
        \begin{itemize}
            \item \textit{Name of interested party:} reason why party is interested
        \end{itemize}

    \item \textbf{Preconditions:}
        \begin{itemize}
            \item First precondition.
            \item Second precondition.
        \end{itemize}

    \item \textbf{Postconditions:}
        \begin{itemize}
            \item First postcondition.
            \item Second postcondition.
        \end{itemize}
        
    \item \textbf{Main scenario:} 
    \begin{enumerate}
       \item Step 1
       \item Step 2
       \item Step 3
       \item \ldots
    \end{enumerate}

    \item \textbf{Alternative scenarios:} 
    \begin{enumerate}
        \item [3b.] Alternative at step 3
    \end{enumerate}
    
    \item \textbf{Remarks:}
        \begin{itemize}
            \item First remark
        \end{itemize}
\end{itemize}

\subsection{\emph{UC1.1}: Consult document status}
\begin{itemize}
    \item \textbf{Name:} Consult document status
    \item \textbf{Primary actor:} Customer organisation
    \item \textbf{Interested parties:} 
        \begin{itemize}
            \item \textit{Customer organisation:} wants to consult the status of one of its documents at eDocs
            \item \textit{The eDocs system:} provides a management dashboard through which the customer can, among other things, consult the status of its documents
        \end{itemize}

    \item \textbf{Preconditions:}
        \begin{itemize}
            \item The customer organisation is a registred client at eDocs
            \item The customer organisation is logged into the eDocs system
        \end{itemize}

    \item \textbf{Postconditions:}
        \begin{itemize}
            \item The customer organisation has consulted the status of the desired document
        \end{itemize}
        
    \item \textbf{Main scenario:} 
    \begin{enumerate}
       \item The customer orgnanisation provides the eDocs system with enough information for it to identify the desired document of which the status needs to be consulted
       \item The eDocs system retrieves the status of the respective document
       \item The eDocs system provides the customer organisation with the retrieved document status information
       \item The customer organisation consults the document status information received from the eDocs system
    \end{enumerate}
\end{itemize}

\subsection{\emph{UC2}: Register customer organisation}
\begin{itemize}
    \item \textbf{Name:} Register customer organisation
    \item \textbf{Summary:} The eDocs administrator registers a new customer organisation in order to enlarge the eDocs client base
    \item \textbf{Primary actor:} eDocs administrator
    \item \textbf{Secondary actor:} Customer organisation
    \item \textbf{Interested parties:} 
        \begin{itemize}
            \item \textit{Customer organisation:} wants to outsource its document processing to eDocs
            \item \textit{eDocs:} wants to gain new clients  customer organisations) for which it can process documents
            \item \textit{eDocs administrator:} deals with the administrational tasks during the registration of customer organisations
        \end{itemize}
    \item \textbf{Preconditions:}
        \begin{itemize}
            \item The customer organisation has actively expressed its interest in eDocs
            \item The eDocs administrator has initiated negotiations about the contract and SLA with the customer organisation
        \end{itemize}
    \item \textbf{Postconditions:}
        \begin{itemize}
            \item eDocs has gained a new client (and friend)
            \item The customer organisation and eDocs administrator have succesfully negotiated a contract and SLA
        \end{itemize}
        
    \item \textbf{Main scenario:} 
    \begin{enumerate}
       \item The customer organisation has expressed its interest in eDocs as its potential document processing partner
       \item The eDocs administrator starts negotiations with the customer organisation about the contract and SLA
       \item Both parties agree upon the terms and conditions specified in the forenamed documents
       \item The customer organisation is now a client of eDocs and has agreed to outsource (part of) its document processing tasks to eDocs
    \end{enumerate}
    \item \textbf{Alternative scenarios:} 
    \begin{enumerate}
        \item [3a.] The two parties do not reach an agreement
        \item [4a.] The customer organisation expresses its intent to outsource its document processing tasks elsewhere
        \item [5a.] eDocs had lost a potential client
    \end{enumerate}
    
    \item \textbf{Assumptions:}
        \begin{itemize}
            \item The registration process is initiated by the customer organisation expressing its interest in eDocs. It is thus assumed that eDocs is a big player in the document processing market, following a passive strategy to gain new clients.
        \end{itemize}
\end{itemize}


\subsection{\emph{UC3}: Deliver raw invoice data}
\begin{itemize}
    \item \textbf{Name:} Deliver raw invoice data
    \item \textbf{Summary:} The customer organisation provides eDocs with the raw invoice data to be processed
    \item \textbf{Primary actor:} Customer organisation
    \item \textbf{Secondary actor:} eDocs system
    \item \textbf{Interested parties:} 
        \begin{itemize}
            \item \textit{Customer organisation:} wants to deliver its raw invoice data to eDocs so that the latter can process it accordingly
            \item \textit{System:} accepts the raw invoice data to be processed
        \end{itemize}
    \item \textbf{Preconditions:}
        \begin{itemize}
            \item The customer organisation is logged in to the eDocs system
        \end{itemize}
    \item \textbf{Postconditions:}
        \begin{itemize}
            \item The customer organisation has provided the eDocs system with its raw invoice data
            \item The eDocs system has succesfully received the customer organisation's raw invoice data
        \end{itemize}
        
    \item \textbf{Main scenario:} 
    \begin{enumerate}
       \item The customer organisation sends its raw invoice data to eDocs via the management dashboard or via a web service that provides a link from the customer organisation's system to that of eDocs
       \item The eDocs system receives the customer organisation's raw invoice data
    \end{enumerate}
\end{itemize}


\subsection{\emph{UC4}: Deliver raw payslip data}
\begin{itemize}
    \item \textbf{Name:} Deliver raw payslip data
    \item \textbf{Summary:} The customer organisation provides eDocs, either directly or indirectly, with the raw payslip data to be processed
    \item \textbf{Primary actor:} Customer organisation
    \item \textbf{Secondary actor:} eDocs system
    \item \textbf{Interested parties:} 
        \begin{itemize}
            \item \textit{Customer organisation:} wants to deliver its raw payslip data to eDocs so that the latter can process it accordingly
            \item \textit{Social secretary:} wants to be an intermediate party in the delivery of raw payslip data
            \item \textit{System:} accepts the raw payslip data to be processed
        \end{itemize}
    \item \textbf{Preconditions:}
        \begin{itemize}
            \item The customer organisation representative is logged in to the eDocs system
        \end{itemize}
    \item \textbf{Postconditions:}
        \begin{itemize}
            \item The customer organisation representative has provided the eDocs system with its raw payslip data
            \item The eDocs system has succesfully received the customer organisation's raw payslip data
        \end{itemize}
        
    \item \textbf{Main scenario:} 
    \begin{enumerate}
       \item The customer organisation sends its raw payslip data to eDocs via the management dashboard or via a web service that provides a link from the customer organisation's system to that of eDocs
       \item The eDocs system receives the customer organisation's raw payslip data
    \end{enumerate}
    \item \textbf{Alternative scenarios:} 
    \begin{enumerate}
        \item [1a.] The customer organisation provides its social secretary with the necessary employee and tax information
        \item [2a.] The social secretary integrates the raw employee data into raw payslip data conform to the eDocs standard
        \item [3a.] Continue from step 2.
    \end{enumerate}
    \item \textbf{Remarks:}
        \begin{itemize}
        	\item The customer organisation representative is either its social secretary or the customer organisation itself
        \end{itemize}
\end{itemize}


\subsection{\emph{UC5}: Process raw data}
\begin{itemize}
    \item \textbf{Name:} Process raw data
    \item \textbf{Summary:} The eDocs system processes the raw data that was provided by the customer organisation in a previous step
    \item \textbf{Primary actor:} eDocs system
    \item \textbf{Secondary actor:} None
    \item \textbf{Interested parties:} 
        \begin{itemize}
            \item \textit{Customer organisation:} wants its raw data to be processed correctly
            \item \textit{eDocs system:} wants to process the raw data correctly
        \end{itemize}
    \item \textbf{Preconditions:}
        \begin{itemize}
            \item The customer organisation, to which the raw data belongs, is a registered client of eDocs
            \item The raw data that has to be processed, was successfully delivered to the eDocs system
        \end{itemize}
    \item \textbf{Postconditions:}
        \begin{itemize}
            \item The raw data is processed
        \end{itemize}
        
    \item \textbf{Main scenario:} 
    \begin{enumerate}
       \item Check raw data for consistency and completeness
       \item Process meta data
       \item Process document data according to meta data and previously provided template
       \item Generate document
    \end{enumerate}
    \item \textbf{Alternative scenarios:} 
    \begin{enumerate}
        \item [2a.] Erroneous data is sent back to the customer organisation from which it originated
        \item [3a.] The customer organisation corrects the erroneous part(s) of the respective (batch of) raw data
        \item [4a.] The customer organisation delivers a (batch of) raw data consisting only of those corrected records that were (partly) erronous at the beginning of this processing phase
        \item [5a.] Restart from step 1.
    \end{enumerate}
    \item \textbf{Remarks:}
        \begin{itemize}
        	\item If the raw data were marked as critical by the provider, they should be processed with the highest priority (within the overall five hour deadline)
        \end{itemize}
\end{itemize}


\subsection{\emph{UC6}: De-activate personal document store}
\begin{itemize}
	\item \textbf{Summary:} A Registered recipient de-activates his or her personal document store account.
    \item \textbf{Primary actor:} Registered recipient
    \item \textbf{Secondary actors:} The System
    \item \textbf{Interested parties:}
        \begin{itemize}
            \item \textit{Registered recipient:} wants to de-activate his or her personal document store
            \item \textit{eDocs:} wants to keep as many Registered recipients as possible.
        \end{itemize}

    \item \textbf{Preconditions:} None

    \item \textbf{Postconditions:}
        \begin{itemize}
            \item The Registered recipient is now an Unregistered recipient and can no longer log in to the system.
            \item The Registered recipient is now logged out and can no longer make use of his/her personal document store.
            \item The System has removed the account of the Registered Recipient. It keeps the documents of the Registered recipient archived.
            \item The Registered recipient now receives his or her documents via another channel than his or her personal document store.
        \end{itemize}
        
    \item \textbf{Main scenario:} 
    \begin{enumerate}
       \item The Registered recipient indicates that he or she wants to de-activate his or her personal document store.
       \item The System removes the account of the registered recipient.
       \item The System indicates success to the Registered recipient and logs him or her out.
    \end{enumerate}

    \item \textbf{Alternative scenarios:} 
    \begin{enumerate}
        \item \ldots
    \end{enumerate}
    
    \item \textbf{Remarks:}
        \begin{itemize}
        	\item None
        \end{itemize}
\end{itemize}


\subsection{\emph{UC7}: Provide document template}
\begin{itemize}
	\item \textbf{Summary:} An eDocs administrator
    \item \textbf{Primary actor:} eDocs administrator
    \item \textbf{Interested parties:} 
        \begin{itemize}
        	\item \textit{eDocs administrator:} wants to provide a document template that a customer organization can fill in.
            \item \textit{Customer organization:} wants to have a template for its documents
            \item \textit{eDocs:} wants to generate documents based on a template that the Customer organization has filled in.
        \end{itemize}

    \item \textbf{Preconditions:}
        \begin{itemize}
            \item The Customer organization has registered itself with eDocs.
        \end{itemize}

    \item \textbf{Postconditions:}
        \begin{itemize}
            \item The System now contains a template which the Customer organization can fill in.
        \end{itemize}
        
    \item \textbf{Main scenario:} 
    \begin{enumerate}
		\item The eDocs administrator provides a document template to the System.
    \end{enumerate}

    \item \textbf{Alternative scenarios:} 
    \begin{enumerate}
    	\item None
    \end{enumerate}
    
    \item \textbf{Remarks:}
        \begin{itemize}
        	\item None
        \end{itemize}
\end{itemize}

\subsection{\emph{UC8}: Update document template}
\begin{itemize}
	\item \textbf{Summary:} The Customer organization updates a document template.
    \item \textbf{Primary actor:} Customer organization
    \item \textbf{Secondary actors:} 
    	\begin{itemize}
        	\item eDocs administrator
        	\item eDocs System
        \end{itemize}eDocs administrator
    \item \textbf{Interested parties:} 
        \begin{itemize}
            \item \textit{Customer organization:} wants its documents to have a layout corresponding to the initial template it fills in.
            \item \textit{eDocs:} wants to generate documents with a layout corresponding to the wishes of its clients.
        \end{itemize}

    \item \textbf{Preconditions:}
        \begin{itemize}
            \item The Customer organization is logged in.
        \end{itemize}

    \item \textbf{Postconditions:}
        \begin{itemize}
            \item The System from now on generates documents using the new filled in document template.
            \item The System stores the provided document template.
        \end{itemize}
        
    \item \textbf{Main scenario:} 
    \begin{enumerate}
       \item The Customer organization indicates it wants to download the current document template.
       \item The System looks up the correct document template and provides it to the Customer organization.
       \item The Customer organization changes the current document template.
       \item The Customer organization indicates it wants to upload the filled in template.
       \item The System downloads the updated document. It notifies an eDocs administrator that a new filled in document is waiting to be verified for correctness.
       \item An eDocs administrator verifies the correctness of the document template.
       \item The System stores it. The updated document replaces the old document template for that Customer organization and that document type.
    \end{enumerate}

    \item \textbf{Alternative scenarios:} 
    \begin{enumerate}
        \item [6a.] If the provided document template is not correctly filled in, the administrator notifies the Customer organization. Continue from step 3.
    \end{enumerate}
    
    \item \textbf{Remarks:}
        \begin{itemize}
            \item If the Customer organization has not filled in a document template before, the document template presented to the Customer organization is the document template that the eDocs administrator provided for the Customer Organization to use.
        \end{itemize}
\end{itemize}

\subsection{\emph{UC1}: Enable receipt tracking}
\begin{itemize}
	\item \textbf{Summary:} The Customer organization enables receipt tracking
    \item \textbf{Primary actor:} Customer organization
    \item \textbf{Interested parties:} 
        \begin{itemize}
            \item \textit{Customer organization:} wants to know if a Recipient has received his or her documents.
            \item \textit{Recipient:} might want to know if he or she is being tracked.
        \end{itemize}

    \item \textbf{Preconditions:}
        \begin{itemize}
            \item The Customer organization is logged in.
        \end{itemize}

    \item \textbf{Postconditions:}
        \begin{itemize}
            \item Receipt tracking is now enabled for the distribution channels that support it.
        \end{itemize}
        
    \item \textbf{Main scenario:} 
    \begin{enumerate}
       \item The Customer organization indicates that it wants to enable receipt tracking.
       \item The System stores this setting.
    \end{enumerate}

    \item \textbf{Alternative scenarios:} 
    \begin{enumerate}
        \item \ldots
    \end{enumerate}
    
    \item \textbf{Remarks:}
        \begin{itemize}
            \item Receipt tracking is only supported for e-mail, Zoomit and the personal document store.
        \end{itemize}
\end{itemize}
\subsection{\emph{UC1}: Disable receipt tracking}
\begin{itemize}
	\item \textbf{Summary:} The Customer organization disables receipt tracking
    \item \textbf{Primary actor:} Customer organization
    \item \textbf{Interested parties:} 
        \begin{itemize}
            \item \textit{Customer organization:} no longer wants to be able to track receipts.
            \item \textit{Recipient:} might want to know if he or she is being tracked.
        \end{itemize}

    \item \textbf{Preconditions:}
        \begin{itemize}
            \item The Customer organization is logged in.
        \end{itemize}

    \item \textbf{Postconditions:}
        \begin{itemize}
            \item Receipt tracking is now disabled for all distribution channels.
        \end{itemize}
        
    \item \textbf{Main scenario:} 
    \begin{enumerate}
       \item The Customer organization indicates that it no longer wants to enable receipt tracking.
       \item The System stores this setting.
    \end{enumerate}

    \item \textbf{Alternative scenarios:} 
    \begin{enumerate}
        \item \ldots
    \end{enumerate}
    
    \item \textbf{Remarks:}
        \begin{itemize}
            \item Receipt tracking is only supported for e-mail, Zoomit and the personal document store.
        \end{itemize}
\end{itemize}

\subsection{\emph{UC1}: Send document via e-mail}
\begin{itemize}
    \item \textbf{Summary:} Initiated by Time, the system sends an e-mail to an Unregistered recipient.
    \item \textbf{Primary actor:} Time
	\item \textbf{Secondary actors:} The System
    \item \textbf{Interested parties:} 
        \begin{itemize}
            \item \textit{Customer organization:} wants the Unregistered recipients to receive their documents.
            \item \textbf{Unregistered recipient:} wants to receive his or her documents.
            \item \textbf{eDocs;} wants the documents it sends to be correctly received.
            \item \textbf{E-mail provider} wants to correctly deliver documents it send by e-mail in a timely manner.
        \end{itemize}

    \item \textbf{Preconditions:}
        \begin{itemize}
            \item The document to be send is generated correctly.
            \item The addressing information provided by the Customer Organization in the raw data is an e-mail address.
        \end{itemize}

    \item \textbf{Postconditions:}
        \begin{itemize}
            \item The System has send out an e-mail to the Unregistered recipient.
        \end{itemize}
        
    \item \textbf{Main scenario:} 
    \begin{enumerate}
       \item Time notifies the System that it should send the generated document.
       \item The System generates a unique link. When followed, the link points to a web page where the whole document can be read.
       \item The System generates an e-mail containing a preview of the document and the unique link  pointing to the web page with the document.
       \item The System sends the e-mail to the e-mail address provided as  addressing information in the raw data. 
       \item The System marks the corresponding document processing job as sent by e-mail.
       \item If this document is not part of a recurring batch of document processing jobs, the System adds the cost of delivering the generated document to the bill of the customer organization.
    \end{enumerate}

    \item \textbf{Alternative scenarios:} 
    \begin{enumerate}
        \item[2a] If receipt tracking is disabled:
        	\begin{enumerate}
        		\item The System generates an e-mail containing a short message indicating that there is a new document from the Customer organization. It puts the document as an attachment to the e-mail.
       			\item Continue at step 4.
       		\end{enumerate} 
        \item [2b.] The unique link is not unique. The System keeps generating new links until it generates a unique one.	
    \end{enumerate}
\end{itemize}

\subsection{\emph{UC1}: Notify about incorrect e-mail address}
\begin{itemize}
    \item \textbf{Summary:} After receiving an error message from an e-mail provider, the System notifies the customer organization about the incorrect addressing information.
    \item \textbf{Primary actor:} E-mail provider
	\item \textbf{Secondary actors:} None
    \item \textbf{Interested parties:} 
        \begin{itemize}
            \item \textit{Customer organization:} wants to be notified when it gives incorrect addressing information to the eDocs System.
            \item \textbf{eDocs:} wants to notify a customer organisation about incorrect addressing information.
            \item \textbf{E-mail provider} wants to deliver e-mails to the correct e-mail addresses.
        \end{itemize}

    \item \textbf{Preconditions:}
        \begin{itemize}
            \item UC: \emph{Send document via e-mail}
            \item The addressing information provided by the Customer Organization in the raw data is an incorrect e-mail address.
        \end{itemize}

    \item \textbf{Postconditions:}
        \begin{itemize}
            \item The Customer organization is notified about the wrong addressing information.
        \end{itemize}
        
    \item \textbf{Main scenario:} 
    \begin{enumerate}
       \item The E-mail provider receives an e-mail for an incorrect e-mail address.
       \item The E-mail provider notifies the System it has received an e-mail to an incorrect address.
       \item The System extracts the necessary information from the error message received from the E-mail provider.
       \item The System looks up the raw data in its archive corresponding to the incorrect e-mail address and document.
       \item The System notifies the Customer organization about the incorrect addressing information.
    \end{enumerate}

\end{itemize}
\subsection{\emph{UC1}: Receive document via e-mail}
\begin{itemize}
    \item \textbf{Summary:} The Unregistered recipient recipient receives a document via e-mail.
    \item \textbf{Primary actor:} Unregistered Recipient
	\item \textbf{Secondary actors:} The System
    \item \textbf{Interested parties:} 
        \begin{itemize}
            \item \textit{Customer organization:} wants the Unregistered recipients to receive their documents.
            \item \textbf{Unregistered recipient:} wants to receive his or her documents.
            \item \textbf{eDocs;} wants the documents it sends to be correctly received.
            \item \textbf{E-mail provider} wants to correctly deliver documents it send by e-mail in a timely manner.
        \end{itemize}

    \item \textbf{Preconditions:}
        \begin{itemize}
            \item UC: \emph{Send document via e-mail}
        \end{itemize}

    \item \textbf{Postconditions:}
        \begin{itemize}
            \item The Unregistered recipient has received his or her document .
        \end{itemize}
        
    \item \textbf{Main scenario:} 
    \begin{enumerate}
       \item The Unregistered recipient checks his or her e-mail.
       \item The Unregistered recipient sees the e-mail send by the System and opens it. 
       \item The Unregistered recipient now sees the preview of the document and a unique link pointing to a web page.
       \item The Unregistered recipient follows the link in the e-mail address.
       \item The System gets a request that the Unregistered recipient wants to view a document using the unique link.
       \item The System responds by sending a web page to the Unregistered recipient where he or she can see or download the document.
       \item The Unregistered recipient now sees the page where he or she can see and download the document.   
       \item The System remembers the fact that the Unregistered recipient sees the document (receipt tracking).
     \end{enumerate}

    \item \textbf{Alternative scenarios:} 
    \begin{enumerate}
        \item [3a.] If receipt tracking is disabled:
        \begin{enumerate}
              \item The Unregistered recipient opens the attachment of the e-mail. 
              \item The Unregistered recipient now sees his or her document. 
        \end{enumerate}
    \end{enumerate}
\end{itemize}

\subsection{\emph{UC1}: Search for document in personal document store}
\begin{itemize}
    \item \textbf{Summary:} Registered recipient searches for a specific document in his or her personal document store. 
    \item \textbf{Primary actor:} Registered Recipient
	\item \textbf{Secondary actors:} None
    \item \textbf{Interested parties:} 
        \begin{itemize}
        	\item \textbf{Registered recipients:} want to find a specific document.
            \item \textit{Customer organization:} wants the specific document to be found.
        \end{itemize}

    \item \textbf{Preconditions:}
        \begin{itemize}
            \item The Registered recipient is logged in (cf. \emph{UC1: Log in}.
        \end{itemize}

    \item \textbf{Postconditions:}
        \begin{itemize}
            \item The Registered recipient has received an overview of all the documents he or she received meet his or her search criteria.
        \end{itemize}
        
    \item \textbf{Main scenario:} 
    \begin{enumerate}
       \item The Registered recipient indicates that he or she wants to search for a specific document, e.g. in the overview of received documents (cf. \emph{UC4: Consult personal document store}.
       \item The System presents to the Registered recipient one or more ways to specify criteria that the document he/she is looking for must meet.
       \item The System searches in its archive and looks up the documents that meet the specified requirements. The System presents the found documents to the Registered recipient.
     \end{enumerate}

    \item \textbf{Alternative scenarios:} 
    \begin{enumerate}
        \item [3a.] If the System does not find any documents that match the criteria, it notifies the Registered recipient of this fact. Continue from step 2.
    \end{enumerate}
\end{itemize}

\subsection{\emph{UC1}: Send document via print \& postal service}
\begin{itemize}
    \item \textbf{Summary:} Initiated by Time, the system sends a document by postal mail to an Unregistered recipient.
    \item \textbf{Primary actor:} Time
	\item \textbf{Secondary actors:} Print \& postal service
    \item \textbf{Interested parties:} 
        \begin{itemize}
            \item \textit{Customer organization:} wants the Unregistered recipients to receive their documents.
            \item \textbf{Unregistered recipient:} wants to receive his or her documents.
            \item \textbf{eDocs:} wants the documents it sends to be correctly received.
            \item \textbf{Print\& postal service:} wants to correctly deliver documents it send by e-mail in a timely manner.
        \end{itemize}

    \item \textbf{Preconditions:}
        \begin{itemize}
            \item The document to be send is generated correctly.
            \item The addressing information provided by the Customer Organization in the raw data is a postal address 
        \end{itemize}

    \item \textbf{Postconditions:}
        \begin{itemize}
            \item The System has send out the document to the Recipient through postal mail.
        \end{itemize}
        
    \item \textbf{Main scenario:} 
    \begin{enumerate}
       \item Time notifies the System that it should send the generated document.
       \item The System sends the document to the Print \& postal service via web services, including the complete postal address of the Unregistered recipient.
       \item The Print \& postal service prints and packages the document and delivers it to the Unregistered recipient via postal mail. 
       \item The System marks the corresponding document processing job as sent by postal mail.
       \item If this document is not part of a recurring batch of document processing jobs, the System adds the cost of delivering the generated document to the bill of the customer organization.
    \end{enumerate}
\end{itemize}



\section{Non-functional requirements}\label{sec:non-functional}
In this section, we model the non-functional requirements for the system in the
form of \emph{quality attribute scenarios}. We provide for each type
(availability, performance and modifiability) one requirement.

\subsection{Availability}
\subsubsection{\emph{Av1}: Expiration of unique link}
The unique link to a specific document in the document store, which was once sent to the customer organisation by e-mail, has expired.

\begin{itemize}
    \item \textbf{Source:} Internal
    \item \textbf{Stimulus:}
        \begin{itemize}
            \item The expected time period, in which the customer organisation client follows the received hyperlink to a document and opens it, has reached a certain limit value
        \end{itemize}

    \item \textbf{Artefact:} the internal procedure of generating unique and time sensitive hyperlinks to documents
    \item \textbf{Environment:} some customer organisation clients take more time than expected to follow a hyperlink to a document
    \item \textbf{Response:}
        \begin{itemize}
            \item This does not lead to the customer organisation client not being able to acces his or her document ever again
            \item This does not increase the customer organisation client's effort necessary to acces the respective document significantly
            \item Prevention
		\begin{itemize}
			\item The expiration date of a hyperlink to a document should lie sufficiently far in the future
		\end{itemize}
            \item Detection
		\begin{itemize}
			\item The eDocs system is notified when a customer organisation client attempts to follow an expired hyperlink to a document
		\end{itemize}
            \item Resolution
		\begin{itemize}
			\item If the respective document is still available within the eDocs system, a new unique hyperlink is generated and sent to the respective customer organisation client as part of the response to his or her first hyperlink request
		\end{itemize}
        \end{itemize}

    \item \textbf{Response measure:}
            \item Prevention
		\begin{itemize}
			\item The life expectancy of a unique document link should cover the acces dates of at least 99\% of all customer organisation clients that receive their documents by link
		\end{itemize}
            \item Detection
		\begin{itemize}
			\item The eDocs system should detect all expired links immediately at acces time
		\end{itemize}
            \item Resolution
		\begin{itemize}
			\item The replacement link should be generated according to the normal procedure and should be given the highest priority, resulting in a maximum delay of 3 hours
		\end{itemize}
\end{itemize}

\subsection{Performance}
\subsubsection{\emph{P1}: raw data transfer }


\begin{itemize}
    \item \textbf{Source:} Customer organisation
    \item \textbf{Stimulus:}
        \begin{itemize}
            \item A registered customer organisation sends (a batch of) raw data to the eDocs system
        \end{itemize}

    \item \textbf{Artifact:} the stimulated artifact
    \item \textbf{Environment:} the condition under which the stimulus occurs
    \item \textbf{Response:}
        \begin{itemize}
            \item Describe how the system should respond to the stimulus.
        \end{itemize}

    \item \textbf{Response measure:}
        \begin{itemize}
            \item Describe how the satisfaction of a response is measured.
        \end{itemize}
\end{itemize}

\subsection{Modifiability}
\subsubsection{\emph{M1}: Customized recipient channels}
Currently, eDocs only supports document retrieval through a limited number of channels. In the future, recipients might want to link their own system to that of eDocs to receive their documents. To achieve this, eDocs will need to provide some kind of API.

\begin{itemize}
    \item \textbf{Source:} Recipients
    \item \textbf{Stimulus:}
        \begin{itemize}
            \item Recipients want to be able to receive their documents through customized channels
        \end{itemize}

    \item \textbf{Artefact:} The document delivery process
    \item \textbf{Environment:} The recipient has chosen to receive documents through his or her own custom delivery channel
    \item \textbf{Response:}
        \begin{itemize}
            \item The recipient is presented with an API to link its customized channel to. This API should satisfy all of the recipient's needs
        \end{itemize}

    \item \textbf{Response measure:}
        \begin{itemize}
            \item At least 99\% of all recipients that have ever chosen the customized channel option are satisfied with the provided API and service. This results in at most 1\% of all those recipients issuing some kind of complaint regarding this matter.
        \end{itemize}
\end{itemize}

\end{document}
